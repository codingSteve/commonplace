\cleardoublepage
{\small

The \emph{Pointers} here are not things to be mulled over and analysed in great depth, they're included to give us the little nudges and course corrections for our actions.

Without action of some kind our plans for improvement simply won't get very far. We need to look at the mammoth task in front of us and take that scary first step (and then the next). 
Taking action isn't always easy and choosing the right move (\S \ref{JVG_ACT}) at the right time (\S \ref{ST_WAIT}) can be stupefyingly hard.
The hope is though that we won't be paralysed and we will make a positive impact on the people and projects which are important to us.

We won't be successful all the time (there are sometimes periods of weeks during which I'm not very successful on most fronts) but by persevering (\S \ref{AR_HABIT} and \ref{JP_STAND}) we can chip away at the habits (and even relationships (\S \ref{LC_COMPANY})) that are slowing us down or even undoing good progress we've already made.

We need to remember what's important to us. Remember how much time we have and how we choose to spend it\llink{http://bit.ly/1OMUOuU}. Marcus often looks at how little life he's been allotted and how little he thinks he has left. Although, as Seneca points out, it's not that we are allocated a short time, it's just that we are wasteful of the time we do have. That's a good reminder for us to just get on with things and not get wrapped up in ourselves too much. 

We also need to remember that people will see how we spend our time and the choices we make. We need to be honest with ourselves about what is important to us and what's not. No matter what we say they will believe what they see us do (and what they see us avoid \S \ref{SZ_LESSON}). 

}
